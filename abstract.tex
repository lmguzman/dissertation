\chapter*{Abstract}
\chaptermark{abstract}
\addcontentsline{toc}{chapter}{Abstract}

As a result of the process of descent with modification, closely related species share many traits. Phylogeny thus provides information that not only needs to be considered when making inter-specific comparisons but which also can be leveraged to gain insight into macroevolutionary questions. Here I develop statistical methods, computational tools, and conceptual ideas to help researchers make more meaningful inferences from phylogenetic comparative data. First, I provide a theoretical foundation of how researchers can use models of trait evolution to test hypotheses related to the long-controversial theory of punctuated equilibrium (PE), which asserts that speciation causes rapid evolution against a backdrop of stasis. I break the hypothesis down into four key elements and argue that combining these conceptually distinct ideas under the single framework of PE is distracting and confusing, and more likely to hinder progress than to spur it. Second, I present a suite of statistical software, written in R, for fitting evolutionary models to phylogenetic data. This represents a major overhaul of the popular \textsc{geiger} package. Third, I develop a novel statistical framework for assessing the absolute fit, or adequacy, of phylogenetic models of trait evolution. To date, most researchers have focused almost exclusively on the relative explanatory power of alternative models, rather than the ability of a model to provide a good explanation for the data on its own terms. I use this approach to evaluate the statistical performance of commonly used trait models on 337 comparative datasets covering three key Angiosperm (``flowering plants'') functional traits. In general, the models we tested often provided poor statistical explanations for the evolution of these traits. This was true for many different groups and at many different scales. Last, I develop a new technique for leveraging taxonomic information to assess and overcome sampling biases in trait datasets. I use this to provide the first estimate of the global distribution of woody and herbaceous plants from a database of 39,313 records and find that the world is likely much woodier than researchers previously thought. It is an exciting time for macroevolutionary research as the scale of data has grown tremendously in the last few years. The work presented here will hopefully help researchers make the most of it.

%%% Local Variables:
%%% TeX-master: "thesis"
%%% TeX-PDF-mode: t
%%% End:
