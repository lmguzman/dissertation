\chapter{Introduction: Models, meanings, and macroevolution}
\label{chap:introduction}

%% Find good quote for introduction
%\begin{quote}
%\singlespacing
%In that Empire, the Art of Cartography attained such Perfection that the map of a single Province occupied the entirety of a City, and the map of the Empire, the entirety of a Province. In time, those Unconscionable Maps no longer satisfied, and the Cartographers Guilds struck a Map of the Empire whose size was that of the Empire, and which coincided point for point with it. The following Generations, who were not so fond of the Study of Cartography as their Forebears had been, saw that that vast map was Useless, and not without some Pitilessness was it, that they delivered it up to the Inclemencies of Sun and Winters. In the Deserts of the West, still today, there are Tattered Ruins of that Map, inhabited by Animals and Beggars; in all the Land there is no other Relic of the Disciplines of Geography
%\end{quote}


\section{Phylogenetics is the new paleobiology}

Investigating similarities and differences across species (the ``comparative method'') has been an esssential conceptual tool in the study of adaptation since \citet{Darwin1859}. Interspecific comparisons are especially valuable when there is little or no variation in the trait of interest within species; in these cases, complimentary approaches such as field experiments are of limited utility. Even when there is variation within a species, the comparative approach allows biologists to assess the generalities of patterns. Phylogenetic comparative methods (PCMs) for the study of adaptation arose out of the recognition that shared evolutionary history can confound statistical comparisons \citep{HarveyPagel1991}. As a result of the process of descent with modifications, closely related species share many traits and trait combinations and therefore individual species cannot be considered independent observations. In the 1980s and early 1990s, a number of highly influential statistical approaches were developed to incorporate phylogeny into interspecific comparisons \citep{Ridley1983, Felsenstein1985, Grafen1989, Maddison1990, HarveyPagel1991, Lynch1991, Pagel1994}. 

While initially controversial \citep[e.g.,][]{Westoby1995}, PCMs have gained near universal acceptance in the ensuing decades, such that today, it is near impossible to publish an interspecific study without considering phylogeny. This victory for phylogenetics is so decisive that researchers have expressed concern that the pendulum has swung too far toward phylogenetic comparative approaches \citep{WeberTREE, Losos2011}. However, I argue that the rise of tree thinking has aided a more profound shift in the types of questions researchers ask with phylogenies; and curiously, this shift seemed to occur without anyone noticing. In their original conception, PCMs were a tool in evolutionary ecology; the seminal book by \citet{HarveyPagel1991} is almost entirely about the study of adaptation. But if one picks up a recent issue of \textsc{systematic biology}, \textsc{evolution}, or \textsc{the american naturalist}, it is striking that both the methodological developments and empirical applications are centered on concepts that are deeply rooted in the paleobiological tradition--such as evolutionary rates, species selection, punctuated equilibrium, and adaptive zones. And as noted by \citet{FitzJohnthesis} and \citet{PennellPE}, these concepts are suddenly uncontroversial, even though the fierce debates over the study of ``macroevolution'' that occured in the 1970s and 1980s \citep{Stanley1975, Gould1980, Charlesworth1982} were never satisfactorily settled. (Understanding how and why this happened seems to me a rich area of inquiry for historians and sociologists of science.) It is as if phylogenetics has held the back door open so that paleobiology could sneak into mainstream evolutionary studies without anyone noticing.

Although the debates over macroevolutionary concepts in decades past were undoubtedly frustrating and often unproductive \citep[for example, see][]{Levinton2001, Gould2002}, at least researchers appreciated what was at stake. As I have argued elsewhere \citep{PennellPE, Pennellpcmbook}, the lack of sound theoretical frameworks has created conceptual confusion and hindered progress--it is often unclear exactly what researchers aim to test when they use PCMs to address macroevolutionary questions. My goal in this essay is to define the macroevolutionary research programme. My definition, which I detail below, places statistical models at the center of the discipline. Joe Felsenstein famously wrote: ``Phylogenies are fundamental to comparative biology; there is no doing it without taking them into account'' \citep[][p. 14]{Felsenstein1985}. In a similar vein, I argue that models are fundamental to macroevolution and there is no way to study it without a model-based perspective. As such, I hope to provide a theoretical context and motivation for the work I have done in this dissertation.

This essay is also, in part, a response to Lieberman and Eldredge's \citeyearpar{LiebermanTREE} charge that in my writings on models of trait evolution \citep[][\textsc{Chapter 2} of this dissertation]{PennellPE} that I failed to adequately define what I meant by macroevolution. In some sense, this complaint was off-base--my co-authors and I included the term in the Glossary!--but I concede that our definition (``the study of proccesses and patterns of evolution that occur at or above the level of species'') was rather vague and non-operational. I aim to rectify this oversight here.

\section{What we talk about when we talk about macroevolution}

As with species concepts \citep[reviewed in][]{CoyneOrr}, how we define macroevolution shapes what questions we ask and how we ask them--developing clear working definitions is not just semantic navel-gazing. The term ``macroevolution'' has a long, and often problematic, history in paleontology. Goldschmidt used it to frame his idea of macromutations (``hopeful monsters'') and Osborne and Cope viewed macroevolution as something outside of regular evolution (microevolution) and advocated borderline-mystical ideas about evolutionary progress.

But the most common definitions of macroevolution all center, in some way, around the origin of discontinuities, whether that be the formation of new species, new higher taxa, or new types of traits (evolutionary novelties). This way of thinking came about as an outgrowth of the ``paleobiological revolution'' in 1970s. Led by a group of young (and iconoclastic) researchers--Stephen Jay Gould, Thomas Schopf, Niles Eldredge, David Raup, Steven Stanley, and later, Jack Sepkoski--sought to bring paleontology out of the museum basement and back to the ``high table'' of evolutionary thought. They conceived of paleontology as a nomothetic (``law-forming'') discipline that could contribute genuine evolutionary insights rather than just a catalogue of failed forms \citep[for a fascinating history of the origin of modern paleontology, see][]{Sepkoskibook}. Crucially, they believed that the fossil record could evolutionary insights that would be missed by studying extant species alone. As the study of the fossil record is inheritenly about recognizing discontinuities--only varieities exist, not variation--the emergence of new forms was a natural starting point in the search for new laws. Here they drew inspiration from G.G. Simpson, particularly his early work \citep[e.g.,][his later writings seemed a retreat back to the evolutionary mainstream]{Simpson1944} who viewed macroevolution as involving ``the  rise and divergence of discontinuous groups'' though he was circumspect as to whether this ``differs in kind or only in degree from microevolution'' (p. 97).

[Dobhansky 1937, delimited macroevolution as being the formation of new subspecies and species]

The new crop of paleontologists were much more convinced that macroevolution was indeed wholly different than microevolution and that the discontinuous groups that mattered were species \citep{Stanley1975, Stanley1979, Gould1980, EldredgeCracraft1980, LiebermanTREE}. Macroevolution became synonymous with the emergence and maintenance of species. In his influential book, \citet{Stanley1979} argued that microevolutionary processes--drift, mutation and selection--had analogs at the species level (phylogenetic drift, directed speciation, and species selection, respectively). Punctuated equilibrium \citep{Eldredge1971, EldredgeGould1972, GouldEldredge1977}, which was a lynchpin of the new macroevolutionary research programme, postulated that most phenotypic change occured during speciation such that understanding speciation was more or less the same as studying trait evolution.

This speciation-centric view of the world is widespread in evolutionary biology \citep{Saetre2013}. And given that the data in phylogenetic comparative studies are usually measurements for individual species (even when data is collected from multiple individuals from a species, this is most often summarized by the mean), it is unsurprising that when phylogenetic biologists...

In my view, macroevolutionary research is best thought of as the study of the \emph{long-term dynamics of evolutionary processes in aggregate}.
This is somewhat distinct from how the term has been used previously but I believe this provides an operational definition that can form the basis for a research program in phylogenetics, evolutionary genetics, and paleobiology. In the following sections, I unpack this definition.

\subsection{The time scale of macroevolution}

I proposed that macroevolution research examines dynamics over long time scales, but this is inherently imprecise as ``long'' is a relative term. Indeed, some of the most important recent developments in both ecology \citep[eco-evo-dynamics; e.g.,][]{Shoener2011} and phylogenetic biology \citep[species tree estimation; e.g.,][]{Maddison1997, Edwards2012} involve the realization that processes once thought to occur at different temporal scales actually do interact. For my purposes, it is sufficient to define ``long'' to be time scales longer than the duration of a specific process of interest. To give concrete examples, consider the time required for:
\begin{itemize}
\item a new mutation to fix in a population
\item a population to reach a new fitness optimum
\item reproductive isolation to form between two diverging lineages
\end{itemize}
These times will of course depend on the genetic architecture, population size, etc. but this is not important here. What is important is that in macroevolutionary studies, we are not likely to catch any specific process ``in action'' nor see the clear signal of events that have recently happened. 

\subsection{Studying evolutionary processes in aggregate}

When we are asking macroevolutionary questions--using phylogenetic data or fossil time series--we are really investigating the statistical dynamics of multiple evolutionary processes \citep{Hunt2006}. Over the time periods we are considering, it makes little sense, either logically or mathematically, to ascribe the dynamics to a single cause--selection or drift, for example. And while some of the models used in macroevolution are mathematically equivalent to population-level models \citep{HansenMartins1996, EstesArnold2007, PennellHarmon}, interpreting these in terms of population-level processes is problematic \citep{Harmon2010, Hansen2012book, PennellPE, Pennellpcmbook}. 

Consider the oldest and most common model of trait evolution, Brownian motion \citep[BM;][]{Edwards1964, Felsenstein1973, Thompson1975, Felsenstein1985}. BM is a simple stochastic process where the trait mean $\overline{z}$ evolves as an unbiased random walk
\begin{equation}\label{eq:bm}
\Delta \overline{z}=\sigma dW
\end{equation}
where $dW$ is a Wiener process following a normal distribution with mean 0 and variance $t$, the time since the process began, and $\sigma$ is the ``rate'' of the process. There are several possible interpretations of this from the standpoint of quantitative genetics \citep{Felsenstein1988, HansenMartins1996} but here I will only consider interpretations that only involve genetic drift. First, Eqn \ref{eq:bm} is equivalent to the change in the population mean expected under pure genetic drift when the additive genetic variance $\mathbf{G}$ is assumed to be constant \citep{Lande1976} such that
\begin{equation}\label{eq:bm-lande} 
\sigma^{\text{2}}=\mathbf{G}/N_e
\end{equation}
where $N_e$ is the effective population size. Alternatively, one can consider the case where $\mathbf{G}$ also changes as a result of drift \citep{LynchHill1986}. In this scenario, the rate of macroevolutionary drift is mutation limited \citep{Felsenstein1988, Lynch1989}
\begin{equation}\label{eq:bm-lynch}
\sigma^{\text{2}}=\text{2}\mathbf{V_m}
\end{equation}
where $\mathbf{V_m}$ is the mutational variance \citep[see][for details]{LynchWalshbook}.

However, neither of these explanations make sense when we examine real comparative data. As an example, I fit a BM model to the wing lengths (a proxy for body size) of the Galapagos finches. The maximum likelihood estimate for the rate parameter $\hat{\sigma}^{\text{2}}=\text{0.070}$. Using my framework for assessing model adequacy (\textsc{Chapter 5} of this dissertation), BM appears to be a statistically adequate explanation for the patterns of the data. Yet, by Eq. \ref{eq:bm-lande} if we assume that $N_e$ is on the order of 10,000, the additive genetic variance would be around 700 [units], which is ridiculous considering that the mean trait value is 4.23. Similarly, by Eq. \ref{eq:bm-lynch}, this would imply that the mutational variance is on the order of 0.035, which is much higher than we would typically expect [refs]. The observation that na\"{i}ve population genetic interpretations of macroevolutionary models are likely incorrect has been made repeatedly in the literature \citep[e.g.,][]{Felsenstein1988, Lynch1990, HansenMartins1996, EstesArnold2007, HohenloheArnold2008, Harmon2010} but is suprisingly poorly appreciated \citep{Pennellpcmbook}. 

So what are we measuring when we use phylogenetic comparative methods. I think the most meaningful answer is that we are modeling the aggregate effect of evolutionary processes. Looking at the world from this perspective can provide insight...

\subsection{Macroevolutionary research questions}

Note that defining the time scale of macroevolution in this way, I am making it clear that some types of evolutionary studies that are not normally considered considered macroevolutionary can fall under this umbrella. Here are three examples:
\begin{enumerate}
\item \emph{Phylogenetic estimation}.
\item \emph{Long-term experimental evolution}
\item \emph{Pattern formation in evolutionary dynamics models}
\end{enumerate}




\section{The central role of models}

The most commonly used models of continuous trait evolution are directly related to models from quantitative genetics. Brownian motion \citep[BM;][]{Edwards1964, Felsenstein1973, Thompson1975, Felsenstein1985} involves 

\section{What about evolutionary novelty?}

We have no conceptual way to think of this. Discuss Gunter Wagner's new stuff. 

\section{Outline of this dissertation}

This dissertation contains five chapters, all but one of which have been published elsewhere. Following this introduction, they all center on the use of models to understand macroevolutionary processes and patterns in a phylogenetic context. In my first chapter, I discuss how models of trait evolution can shed light on the decades-old debate over punctuated equilibrium. My second chapter is about the mechanics of actually fitting evolutionary models to phylogenies. Software development was a large component of my dissertation research and the revamp of the geiger package was the culmination of much of this work. In my third chapter, I use models of trait evolution to examine the evolution of sex chromosomes across vertebrates. The emphasis here is on how phylogenetic models can provide complementary insights to those provided by population-genetic models of evolution. 

In the fourth and fifth chapters, I address outstanding statistical challenges in using trait evolutionary models to make inferences from phylogenetic comparative data. First, I develop a general statistical framework to assess the absolute fit, or adequacy, of phylogenetic models. To date, the focus of the field has been almost exclusively on measuring the relative fit of alternative models without consideration as to whether any of the candidate models actually have explanatory power for the data. Second, I develop a novel method for assessing and correcting for biased sampling in the trait data itself. The fact that data (especially when collated from large databases) is not a random sample of all species has been given short shrift when applying comparative methods though this generate substantial bias in parameter estimation and the fit of macroevolutionary models. 

Taken together, I believe these five chapters have made a significant contribution to how we analyze phylogenetic comparative data and how we think about the results...

\section{Concluding remarks}
Even to those of us accustomed to measuring moments in millions of years, it is impossible to fathom just how deep ``deep time'' really is. We recognize that natural selection is ``daily and hourly scrutinising'' \citep[][p. 84]{Darwin1859} forms but get tripped up when we think about how many days and how many hours the history of life actually contains. We talk of ``rapid'' and ``recent'' radiations that have played out over tens of millions of years as if they just happened yesterday and species are just getting used to their new digs. This is of course understandable -- humans have poor intuition when confronted with things outside of our normal frame of reference...

   

%%% Local Variables:
%%% TeX-master: "thesis"
%%% TeX-PDF-mode: t
%%% End:
