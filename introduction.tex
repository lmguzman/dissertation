\chapter[Introduction: Models, meanings, and macroevolution]{Introduction: Models, meanings, and macroevolution \footnote {This chapter was previously published in a modified form as: Pennell, M.W. 2015. Modern Phylogenetic Comparative Methods and Their Application in Evolutionary Biology: Concepts and Practice--Book Review. Systematic Biology 64:161--163}}
\label{chap:introduction}

\section{Objectives and structure of this dissertation}

The primary goal of my dissertation is to improve the statistical and conceptual foundations that underlie phylogenetic tests of macroevolutionary hypotheses. To address the statistical component, I have: led the development of statistical software for fitting models and analyzing data (\spacedsmallcaps{\Chapref{geiger}}), which I used to study the macroevolution of sex chromosomes across vertebrates (\spacedsmallcaps{Chapter 4}); created a novel framework for assessing the  absolute fit, or adequacy, of models of trait evolution (\spacedsmallcaps{Chapter 5}); and developed a new approach for evaluating and dealing with sampling biases in comparative datasets (\spacedsmallcaps{Chapter 6}), a problem that is becoming increasingly pertinent. if understudied, as researchers rely more on curated collections of data. Additionally, I have been involved in a number of other projects during my Ph.D. (not included in this dissertation) in which I have examined the statistical properties of existing phylogenetic comparative approaches \citep{Pennell2012, UyedaPCA, Sarverprior}, developed new ones \citep{SlaterPennell, ksi} and applied these to test empirical questions in angiosperms \citep{ksi, nestedradiations} and ascidians \citep{Maliska2013}.

As mentioned above, I have coupled my work in statistical methods with more theoretical work, in which I aimed to provide roadmaps for better interpreting the results of tests using these methods \citep{Rosenblum2012, PennellHarmon, PennellPE, PennellTREEresponse, Pennellpcmbook}. One of these projects, an investigation into tests of the theory of punctuated equilibrium \citep{Eldredge1971, EldredgeGould1972} is included as a chapter in this dissertation (\spacedsmallcaps{Chapter 2}). 

As an introduction to my dissertation, I briefly overview the field of comparative biology \citep[for a more comprehensive discussion, see][]{PennellHarmon} and point out what I believe is the biggest challenge facing the field--that we often have a poor understanding of precisely what we are measuring and explaining when we use phylogenetic comparative methods (PCMs)--and highlight possible ways forward.

\section{Overview of phylogenetic comparative methods}

Investigating similarities and differences across species (the ``comparative method'') has been an esssential conceptual tool in the study of adaptation since \citet{Darwin1859}. Interspecific comparisons are especially valuable when there is little or no variation in the trait of interest within species; in these cases, complimentary approaches such as field experiments are of limited utility. Even when there is variation within a species, the comparative approach allows biologists to assess the generalities of patterns. PCMs for the study of adaptation arose out of the recognition that shared evolutionary history can confound statistical comparisons \citep{HarveyPagel1991}. As a result of the process of descent with modifications, closely related species share many traits and trait combinations and therefore individual species cannot be considered independent observations. In the 1980s and early 1990s, a number of highly influential statistical approaches were developed to incorporate phylogeny into interspecific comparisons \citep{Ridley1983, Felsenstein1985, Grafen1989, Maddison1990, HarveyPagel1991, Lynch1991, Pagel1994}. 

While initially controversial \citep[e.g.,][]{Westoby1995}, PCMs have gained near universal acceptance in the ensuing decades, such that today, it is near impossible to publish an interspecific study without considering phylogeny. This victory for phylogenetics is so decisive that researchers have expressed concern that the pendulum has swung too far toward phylogenetic approaches \citep{Losos2011} in the study of evolutionary ecology. While PCMs are still routinely used to test for adaptation, the field has evolved in subtle yet substantial ways: researchers recognized that the same models developed for comparative questions could also be used to test macroevolutionary questions--for example, what is the pattern of trait change through deep time and what processes drove these trends?--that were long the exclusive domain of paleobiology \citep{HansenMartins1996, Hansen1997, Schluter1997, Pagel1997, MooersSchluter1998, Pagel1999, Mooers1999}. The rate of development of novel PCMs has been incredible and this pace has been matched by the ever-increasing availability of more reliable phylogenetic trees along with large-scale efforts to aggregate phenotypic data from across the Tree of Life.

\section{Current challenges in comparative biology}

Despite the incredible progress of phylogenetic comparative methods over the last few decades, there remain some fundamental issues that are deeply unsettling: while we have sophisticated machinery for fitting many different types of models to comparative data, we often lack a clear interpretation of what exactly they mean. Reading many papers in the field (including my own!), I cannot help but recall a sentiment expressed by \citet{Houle2011} in their lucid review of measurement theory and its applications in biology. They criticize statisticians who advocate that data transformations are justifiable whenever they result in distributions that meet the assumptions of a particular analysis: ``If that is statistics, we want no part of it, as science is about nature, not numbers'' [p. 18]. I argue that our ability to analyze phylogenetic comparative data has outpaced our ability to understand it.

Consider for example, regression models of the form
\[Y=\beta_0 + \beta_1X + \epsilon.\]
In phylogenetic regression \citep{Grafen1989, Lynch1991}, it is usually assumed that the tree only enters into the model in the error term $\epsilon$ such that $\epsilon \sim \mathcal{N}(\mathbf{0}, \mathbf{V})$ where $\mathbf{V}$ is the expected variance-covariance matrix for the traits given an evolutionary model. In other words, the evolutionary model is used to model the structure of the residuals and not the actual traits. Formulating the model in such a way allows us to make use of well-established statistical theory from generalized least squares (GLS) and generalized linear mixed-effects (GLM) models \citep{Lynch1991, Rohlf2001, Rohlf2006, Hadfield2010}. Including the phylogenetic structure in the error variance is no different from including any other type of covariance. By recognizing this equivalency, we can now fit phylogenetic regression models with a variety of distributions for the response variable $Y$ \citep{Ives2010, Hadfield2010}, incorporate measurement error \citep{Ives2007, Hansen2012SysBio}, and take advantage of a large number of other standard statistical tricks \citep[see][for a recent review]{PCM}

There are a variety of different models one can use to create $\mathbf{V}$. The most popular is to assume that the residuals are distributed according to the expectations of a Brownian motion (BM) model. Indeed, the original independent contrasts method \citep{Felsenstein1985} produces identical results to a phylogenetic regression model when this assumption is made \citep{Blomberg2012}. A number of the researchers have advocated that a $\lambda$ tree transformation \citep{Pagel1999, Freckleton2002, Freckleton2011} is often more appropriate than simply assuming BM for constructing the error variance term $\mathbf{V}$. (The $\lambda$ transformation involves multiplying the off-diagonals of $\mathbf{V}$ by a estimated parameter between 0 and 1.) This is a purely phenomenological construct--by shrinking every branch except those leading to the tips, it implies that there is something special about extant taxa, which is clearly not the case. Nonetheless, researchers often use such models to claim that one trait is adapted to the value of another. In a series of papers, Hansen and colleagues have clearly articulated the problem with such inferences \citep{HansenOrzack2005, Hansen2008, Labra2009, Hansen2012SysBio}. Effectively, standard regression models assume adaptation to a new environment is instantaneous and that maladaption is phylogenetically structured; closely related species will have similar deviations from the optimal trait value even if the optimum differs between them. From a biological perspective, this seems very odd.

Perhaps even more confusing is the use of Ornstein-Uhlenbeck (OU) models to construct the error variance term. OU is attractive for modeling the residual variance because, unlike the $\lambda$ transformation, it is a coherent stochastic process and is directly analagous to a population level model from quantitative genetics--quadratic stabilizing selection on a fixed adaptive landscape \citep{Lande1976, HansenMartins1996}. While the $\lambda$ transformation is obviously just a statistical construct, OU \emph{seems} biologically motivated. Indeed, researchers commonly interpret the OU-structured variance term as representing stabilizing selection or constraints. But these does not get around Hansen's criticisms. These models still assume phylogenetically structured maladapation and they do they allow researchers to make specific inference about stabilizing selection or evolutionary constraints--it is completely unclear precisely what is being constrained or how the residuals are under stabilizing selection. OU error structures may often fit data better than BM error structures but it is likely that this is simply because OU can accomodate more variance towards the tips of the phylogeny than a BM model can (including $\lambda$ has a similar effect). The evolutionary argument here seems merely window dressing for a purely statistical argument. 

The arguments I have made here apply equally well to models without predictor variables--where what we want to explain with comparative methods are the distribution of traits through time without considering predictor variables. It is now a common exercise in both phylogenetic comparative biology and paleobiology to compare alternative models of trait evolution and then to interpret the best-fitting model in terms of evolutionary processes \citep[e.g.,][]{Mooers1999, Hunt2007, Harmon2010, Hopkins2012, Burbrink2012, Hunt2012}

\section{Paths forward}

How then are we to make sense of comparative analyses? In my view, there are three possible frameworks with which to think about comparative biology. First, we can take the view that what we are measuring are strictly patterns and that we are not necessarily making inferences about specific evolutionary processes. This is certainly a defensible position: the patterns may be interesting in and of themselves and documenting commonalities and differences among clades and through time may provide a broader picture of the history of life on earth. In practice, this is what researchers are often actually doing, even if they are hesitant to admit this. And since the models we used in comparative biology predict trait distributions that conform to common probability distributions, there are undoubtedly a huge number of processes that could generate the patterns we observe \citep{Jaynes2003, Frank2009, Frank2014}.
A benefit of openly adopting this perspecitve is that we can consider a much broader suite of models that may provide a much better fit to our data and predictive power than current models--if we are not interested in making specific evolutionary inferences, then we need not be beholden to specific evolutionary models. Such alternatives may include macroevolutionary diffusion processes \citep[e.g.,][]{Clauset2008}, models derived from macroecological theories \citep{Brown2004, Harte2011} or making use of statistical learning approaches divorced from any process whatsoever.

The second framework is the quantitative genetics view: the models we fit in comparative biology should be taken as literally representing microevolutionary hypotheses. Many of the commonly used models can be directly interpreted in terms of population-level parameters \citep{HansenMartins1996, PennellHarmon}. We can compare the estimated model parameters to within-population measures to test if macroevolutionary divergences are consistent with evolution by drift, stabilizng selection, etc. This project is certainly interesting and worth pursuing. But given the results of studies that have explicitly examined this connection \citep{Lynch1990, EstesArnold2007, Hohenlohe2008, Harmon2010, Bolstad2014} using rather simple models, it appears that translating the parameters estimated from comparative data to the terms of quantitative genetics (e.g., if we assume that BM is strictly a model of drift with fixed additive genetic variance $\mathbf{G}$, the estimated rate parameter $\sigma^2$ is equal to $\mathbf{G}$ divided by the effective population size $N_e$ ; \citealt{Lande1976}) will often result in nonsensical numbers. 

The third perspective is to take seriously the idea that macroevolutionary models reflect the dynamics of adaptive landscapes through deep time \citep{Arnoldetal2001, Hansen2012book, PennellPE}. This is in line with the views of Hansen, O'Meara/Beaulieu and Ingram/Mahler.  Comparative biologists have a tendency to discuss many of these ideas in scare quotes. The optimum of OU models is referred to as ``clade level optimum''. A model with decelerating rates of change depicts an ``early burst''. I argue that a much richer and more meaningful connection can potentially be made. Theoretical work over the last century has produced a beautiful and fairly comprehesive understanding of how populations move across adaptive landscapes and empricists have tested the theoretical predictions in a wide variety of systems and contexts. In contrast, we have only a preliminary understanding of how the landscapes themselves evolve on longer time scales. This is a fundamentally important question in evolutionary biology and one which I believe, phylogenetic comparative biology and paleobiology can help address. 

There is a lot of work to be done before we will really able to get at these types of questions. Once we recognize that some of the classic concepts in evolutionary biology--such as adaptive zones, adaptive radiations and key innovations--are actually hypotheses about the structure and dynamics of adaptive landscapes \citep{Hansen2012book}, we can start developing statistical models that actually capture their essential properties. Current models are, at best, loosely tied to these ideas (hence the scare quotes). Additionally, there are a number of exisiting mathematical frameworks that make predictions about these higher order processes and trait evolution over longer time periods \citep[see for example,][]{Holt2003, Gavrilets2004, Doebeli2011}. But there is currently no way to estimate the relevant parameters of these models from comparative data.  

\section{Concluding remarks}

Both the development of new PCMs and the interest in using them has grown tremendously over the last decade.  Nevertheless, I feel that we, as a field, are somewhat stuck. First, the same handful of statistical models are employed over and over again with most of the progress representing relatively minor variations on similar themes. (That is not the say that such improvements are not challenging or worthwhile; indeed the majority of my dissertation is aimed in precisely this direction.) Second, we are often much too vague on what exactly we want to explain with PCMs--this is apparent in both this current collection and in the literature more broadly. I argue that these two problems are deeply intertwined. The standard collection of models available today, namely those based on BM and OU, have had such staying power in part because they can be useful for detecting patterns, can be interpreted in light of evolutionary genetics and can loosely be tied to questions about adaptive landscapes. Requiring this sort of conceptual flexibility is also a limitation. More focused, question-specific approaches to modeling that are directly tied to the inferences we actually want to make will likely get us much further than sticking to models that are more general but address no questions particularly well.

 

   

%%% Local Variables:
%%% TeX-master: "thesis"
%%% TeX-PDF-mode: t
%%% End:
