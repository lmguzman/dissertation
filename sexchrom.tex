\chapter[Y fuse? Using phylogenetic and population genetic models to understand sex chromosome fusions]{Y fuse? Using phylogenetic and population genetic models to understand sex chromosome fusions \footnote{In press as: Pennell M.W., Kirkpatrick M., Otto S.P., Vamosi J.C., Piechel C.L., Valenzuela N., and Kitano J. Y fuse? Sex chromosome fusions in fishes and reptiles. PLoS Genetics.}}
\label{chap:sexchrom}

\section{Summary}

The evolution of chromosome number plays an important role in divergent adaptation and speciation. Chromosomal fusion is a common mechanism of karyotypic evolution, but there is little understanding of the evolutionary forces that have driven chromosomal fusions. Because sex chromosomes (X and Y in male heterogametic systems, Z and W in female heterogametic systems) differ in their selective, mutational, and demographic environments, those differences provide a unique opportunity to dissect the evolutionary forces that drive chromosomal fusions. We estimate the rate at which fusions between sex chromosomes and autosomes establish across the phylogenies of both fishes and squamate reptiles. Both the incidence among extant species and the establishment rate of Y-autosome fusions is much higher than for X-autosome, Z-autosome, or W-autosome fusions. Using population genetic models, we show that this pattern cannot be reconciled with many standard explanations for the spread of fusions. In particular, direct selection acting on fusions or sexually antagonistic selection cannot, on their own, account for the predominance of Y fusions. We identify three plausible explanations for the excess of Y-autosome fusions: (i) fusions are deleterious, and the mutation rate is male-biased or the reproductive sex ratio is female-biased, (ii) fusions capture loci under sexually antagonistic selection, and the mutation rate is male-biased or the reproductive sex ratio is female-biased, and (iii) meiotic drive acts against fusions in females. These results may shed light on the processes that drive structural changes throughout the genome.

\section{Introduction}

The number of chromosomes is one of the most fundamental features of a eukaryotic genome. Chromosome number often varies within species or between closely related species, and such variation can contribute to divergent adaptation and speciation \citep{White1973, King1993, PerezOrtin2002, Chang2013, Hou2014}. Although genetic drift, selection for reduced recombination, and meiotic drive are hypothesized to fix chromosomal fusions \citep{Nachman1995, Guerrero2014}, we have an incomplete understanding of the evolutionary forces that allow fusions and fissions to become established.

Sex chromosomes offer a unique opportunity to dissect these forces. The X and Y chromosomes of male-heterogametic species (as in mammals) and the Z and W chromosomes of female-heterogametic species (as in birds) differ in many aspects of their evolutionary environments, particularly with respect to hemizygosity (i.e., XX and ZZ individuals are common, but not YY and WW). While Y and W chromosomes are often thought to be evolutionarily similar, they differ in the amount of time spent in males and females: Y chromosomes spend 100\% of their evolutionary history in males, while W chromosomes spend none. X and Z chromosomes also differ: X chromosomes spend 1/3 of their evolutionary history in males, while Z chromosomes spend 2/3 of their history in males. Consequently, the four types of sex chromosomes vary in how selection acts on them, in their effective population sizes, in their mutation rates, and in the relative importance of meiotic drive \citep{Ellegren2011, Bachtrog2011, Perrinbook}. All of these factors could play a role in the evolution of chromosomal rearrangements, and so differences in rates of rearrangement among sex chromosomes offer clues to what evolutionary conditions favor changes to genome structure.

Structurally, sex chromosomes are the most rapidly evolving parts of the genome in many groups of animals \citep{White1973, Bull1983, Ezaz2006,Perrinbook} In some taxa, such as fishes and squamate reptiles, closely related species (and even populations within a species) differ in how sex is determined \citep{Ezaz2006, Bachtrog2014}. Further, a large number of fusions between sex chromosomes and autosomes have been discovered \citep{White1973, ToS}. Thus there are many phylogenetically independent events, providing the opportunity to test whether fusions involving the four different types of sex chromosomes are equally likely to occur and/or establish within a species.
 
A fusion between a sex chromosome and an autosome can usually be detected because it creates an odd number of chromosomes in one sex \citep[Figure \ref{fig:fuse-diag};][]{Ohno1967,White1973}. With XY sex determination, a Y-autosome fusion creates an X1X2Y system, with the unfused homologue segregating as a neo-X chromosome. Likewise, X-autosome fusions generate XY1Y2 systems, Z-autosome fusions generate ZW1W2 systems, and W-autosome fusions generate Z1Z2W systems. These neo-sex chromosome systems can often be identified by light microscopy, without molecular cloning or linkage mapping. This has enabled cytogenetic studies to identify many species with sex chromosome-autosome fusions \citep{White1973, Ezaz2009, Kitano2012, Yoshida2012, Maddison2013}. These data have yet to be used to estimate rates of different types of sex-autosome fusions.

\begin{figure}
\centering
\includegraphics[width=\textwidth]{figs/Fuse-Fig1}
\caption[Schematic of sex chromosome fusions]{Sex chromosome-autosome fusions create multiple sex chromosome systems. (A) In XY systems, X-autosome and Y-autosome fusions make XY1Y2 and X1X2Y systems, respectively. (B) In ZW systems, Z-autosome and W-autosome fusions make ZW1W2 and Z1Z2W systems, respectively.}
\label{fig:fuse-diag}
\end{figure}

Three main evolutionary forces have been thought to be important to the establishment of fusions. The first is direct selection. While chromosome rearrangements are often considered deleterious \citep{King1993, Gardner2012}, chromosomal translocations may alter the expression of genes near the breakpoint \citep{Ohno1967, Dobigny2004}, which may sometimes be beneficial \citep{PerezOrtin2002, Chang2013}. A second mechanism that has been proposed to establish fusions is sexually antagonistic selection at an autosomal locus \citep{Charlesworth1982}. A fusion with a sex chromosome can cause an allele that is beneficial in one sex to spend more than half of its evolutionary life in that sex. Meiotic drive is a third force. During female meiosis in animals, one of the products of meiosis goes into the egg, while the others are discarded in the polar bodies. In some species, female meiotic drive preferentially transmits fused chromosomes to eggs, while unfused chromosomes go into polar bodies \citep{Pardo2001a, Pardo2001b}. This drive favors X-autosome fusions because they experience female meiosis in two of every three generations. In other species, female meiotic drive preferentially transmits fused chromosomes, which should select against X-autosome fusions \citep{Yoshida2012}. While these evolutionary forces are known to affect the spread of sex chromosome-autosome fusions, previous work has not examined the relative rates at which fusions with different types of sex chromosomes establish within a population.
 
We begin this study by analyzing a large new data set that includes information on the sex determination system and karyotypes across the tree of life \citep{ToS}. We focus on fishes and squamate reptiles because these taxa include many independent origins of XY and ZW systems \citep{Ezaz2009, Kitano2012}, allowing us to assess differences in the rates of fusions. We find that Y-autosome fusions fix at a much higher rate than any of the other three types of sex chromosome-autosome fusions. This then motivates us to develop an integrated body of analytic models that predict the relative fixation rates for the different types of fusions. The models incorporate a large number of potentially important factors: deleterious and beneficial effects of fusions, sexually antagonistic selection, female meiotic drive, genetic drift, sex-biased mutation rates, and biased sex ratios. We find that several of the data cannot be explained by some of the most frequently-discussed hypotheses. There are, however, several combinations of forces that are able account for the observed patterns of sex chromosomes fusions, as we highlight.

\section{Analysis of patterns of sex chromosome-autosome fusions in vertebrates}

We compiled lists of species with multiple sex chromosome systems (X1X2Y, XY1Y2, ZW1W2, and Z1Z2W systems) from the Tree of Sex database \citep{ToS}. Although X1X2Y systems (or ZW1W2 systems) can also arise from species with XO (or ZO) systems through a reciprocal translocation between an X (or a Z) and an autosome \citep{White1973, Kitano2012}, XO or ZO systems are rare in vertebrates \citep{ToS}. In addition, although fission of sex chromosomes can also create multiple sex chromosome systems \citep{White1973, Kitano2012}, such fissions are also rare in vertebrates \citep{Ohno1967, Kitano2012, Yoshida2012}. Therefore, we considered that most multiple sex chromosome systems are derived from sex chromosome-autosome fusions in vertebrates. We address two questions with our empirical analyses. First, do Y-A (W-A) fusions occur at different rates than X-A (Z-A) fusions? Second, are there differences in rates of fusion between male and female heterogametic lineages? For both questions, we first simply tabulated the numbers in the database and computed Fisher's exact test. This ignores phylogenetic non-independence but allowed us to use all of the available data. 

Examining the raw counts (Table \ref{tab:fusions}), two interesting patterns emerge. Hereafter, we refer to the fusion between a Y chromosome and an autosome as Y-A fusion, and similarily for other sex chromosomes. First, there are more species with Y-A fusions (X1X2Y karyotype, 101 species) than with X-A fusions (XY1Y2 karyotype, 27 species). The pattern is particularly strong in both fishes and squamate reptiles, while the numbers are more nearly equal in mammals (Table \ref{tab:fusions}). Such counts, however, do not account for the phylogenetic relatedness among many of the species. Second, sex chromosomes in XY lineages are more often fused than those in ZW lineages (Table 1). In fishes, 41.3\% (45/109) of XY species have fused sex chromosomes, whereas only 5.3\% (2/38) of ZW species do (Fisher's exact test $p < \text{0.001}$). In reptiles, 33\% (40/120) of XY species have fusions, whereas only 2.5\% of species (6/240) of ZW species do (Fisher's exact test $p < \text{0.001}$).

\begin{table}
\centering
\begin{tabular}{|p{.15\textwidth} |p{.1\textwidth} |p{.1\textwidth}|p{.1\textwidth}|p{.1\textwidth}|p{.1\textwidth}|p{.1\textwidth} |}
\hline
Taxa & Y-A (X1X2Y) & X-A (XY1Y2) & W-A (Z1Z2W) & Z-A (ZW1W2) & XY systems & ZW systems \\ \hline
Fish & 42 & 3 & 0 & 2 & 109 & 38 \\\hline
Amphibians & 1 & 0 & 0 & 0 & 29 & 16 \\\hline
Reptiles & 40 & 0 & 2 & 4 & 120 & 240 \\\hline
Birds & - & - & 2 & 4 & 0 & 192\\\hline
Mammals & 18 & 24 & - & - & 467 & 0\\\hline
\end{tabular}
\caption[Records of sex chromosome-autosome fusions in vertebrates]{Observed number of species with multiple sex chromosome systems in vertebrates. Only X1X2Y, XY1Y2, Z1Z2W, and ZW1W2 systems are counted here.}
\label{tab:fusions}
\end{table}
   

To gain a better estimate of the rates at which fusions establish with different chromosomes, we fit phylogenetic models to the fusion data. We first matched sex chromosome systems from the fish dataset to a recent time-calibrated phylogeny of teleosts \citep{Rabosky2013}, containing 7811 species (we note that a small number of species were removed from the published phylogeny due to errors discovered after publication; M. Alfaro, personal communication). We matched the data of sex chromosome systems from squamates to the squamate phylogeny \citep{Pyron2013, PyronBurbrink2013} using genetic data from 4161 species. In order to maximize overlap between the trait data and the species, we used an approximate matching algorithm for unmatched species: 1) retain all species that occur in both the tree and the dataset; 2) replace an unmatched species in the tree with a randomly selected unmatched species in the dataset from the same genus as long as this did not result in more than two representatives from the genus (this assumes monophyly of genera but avoids determining node order for nodes not in the original trees). We then pruned down the phylogeny down to those tips with data assignments.This resulted in phylogenetic comparative datasets containing 163 species of fish (Figure \ref{fig:phylo-fish}) and 261 squamate (Figure \ref{fig:phylo-squa}) species.

\begin{figure}
\centering
\includegraphics[width=\textwidth]{figs/Fuse-Fig2}
\caption[Phylogenetic distribution of sex chromosome fusions in fish]{Sex chromosome fusions (outer circle) and sexual determination system (inner circle) mapped onto the phylogenetic trees of fish. The vast majority of fusions occur in XY systems (aqua) and involve Y-A fusions (brown).}
\label{fig:phylo-fish}
\end{figure}

\begin{figure}
\centering
\includegraphics[width=\textwidth]{figs/Fuse-Fig3}
\caption[Phylogenetic distribution of sex chromosome fusions in squamates]{Sex chromosome fusions (outer circle) and sexual determination system (inner circle) mapped onto the phylogenetic trees of squamate reptiles. The vast majority of fusions occur in XY systems (aqua) and involve Y-A fusions (brown).}
\label{fig:phylo-squa}
\end{figure}



We conducted two separate types of phylogenetic analyses on both groups. First, we examined differences between XY and ZW systems; here, we treat X-autosome and Y-autosome fusions as equivalent (and likewise, Z-autosome and W-autosome fusions). Second, we investigated autosomal fusion rates for all types of sex chromosomes individually (i.e., Y-, X-, W-, and Z-autosome fusions). While the second analysis provides more detailed resolution, some of the states are rarely observed (and in some cases, not at all). All analyses were performed using the R package \textsc{diversitree} \citep{FitzJohn2012}, and code to reproduce all results can be found at \url{https://github.com/mwpennell/fuse/analysis}. 

\subsection{Fusion rates in XY vs. ZW systems}

Using a Markov model \citep{Pagel1994}, we considered transitions among the following states:
\begin{enumerate}
\item $XY$: Male heterogametic unfused
\item $XY_F$: Male heterogametic fused (X1X2Y or XY1Y2)
\item $ZW$: Female heterogametic unfused
\item $ZW_F$: Female heterogametic fused (Z1Z2W or ZW1W2)
\end{enumerate}
allowing transitions between all states with $q_{A.B}$ representing the transition rate between states $A$ and $B$. We then used likelihood ratio tests to restrict the model in order to improve our ability to estimate the parameters of interest. 

We first imposed the biologically reasonable constraint that prior to becoming $XY_F$ (or $ZW_F$), a lineage must first be $XY$ (or $ZW$); e.g., the transition rate from female heterogametic unfused to male heterogametic fused $q_{ZW.XY_F}$ would be zero. These restrictions did not lead to a significant decline in likelihood for either squamates or fish and was accepted. Next, we proposed a model in which the rate of switching the heterogametic sex, going from a XY to a ZW system and \emph{vice versa}, did not depend on whether the lineage contained a fused sex chromosome or not (e.g., $q_{XY_F.ZW} = q_{XY.ZW}$). In both fish and squamates, this restriction was acceptable.

In the next step, we proposed a model in which the rate of chromosomal fission, going from a fused sex chromosome system to an unfused system of the same type, was the same for XY and ZW systems. In fish, a likelihood ratio test favored the more restricted model, whereas in squamates, the more general model (where $q_{XY_F.XY} \neq q_{ZW_F.ZW}$) was favored ($p=\text{0.012}$). The support for the more general model in squamates stems from the scarcity of ZW fusions in the data; there is little information to reliably estimate the transition rate from fused female heterogametic to unfused female heterogametic ($q_{ZW_F.ZW}$) using maximum likelihood (see below). We therefore took slightly different approaches when analyzing the two clades.

For fish, we compared the resulting model ($q_{XY_F.XY} = q_{ZW_F.ZW}$, $q_{ZW.XY_F}=q_{XY.ZW_F}=\text{0}$, $q_{XY_F.ZW}=q_{XY_Z.ZW}$, $q_{ZW_F.XY}=q_{ZW.XY}$) to an even more reduced model in which the XY and ZW fusion rates were set to be equal ($q_{XY.XY_F}=q_{ZW.ZW_F}$). We found the rate difference to be highly significant ($p=\text{0.014}$) using a likelihood ratio test. To better accomodate uncertainty in the estimate, we ran a Bayesian analysis. We set broad exponential priors on all parameters ($\lambda=\text{20}$) and sampled 50,000 generations of the MCMC, discarding the first 10,000 as burnin. This also supported our conclusion that XY fusions occur at a higher rate than ZW fusions (98.6\% of the posterior probability supported this and the 95\% credibility interval for the difference in rates did not overlap with zero; Figure \ref{fig:pp-fuse-final}).

\begin{figure}
\centering
\includegraphics[width=\textwidth]{figs/fusion-xyzw.pdf}
\caption[Fusion rate difference between XY and ZW systems]{Posterior probability density of the difference in fixation rates of fusions between autosomes and sex chromosomes (rates in XY species minus in ZW species), for both fish (blue) and squamates (red). The plot illustrates the difference in fusion rates over the last 40,000 steps of an MCMC chain, with the 95\% credibility intervals shown by the horizontal bars below the figure.}
\label{fig:pp-fuse-final}
\end{figure}

For the squamate data, we took two approaches. First, we assumed that the `equal fission rates model' was indeed reasonable and performed the same analysis as in fish. Using a likelihood ratio test, the difference in fusion rates for XY and ZW was found to be highly significant ($p=\text{0.003}$). The same was true for the Bayesian analysis (99.9\% of the posterior probability distribution supported this conclusion; Figure \ref{fig:pp-fuse-final}). Second, we used a Bayesian MCMC to fit a model in which the fission rate $q_{ZW_F.ZW}$ was estimated independently of $q_{XY_F.XY}$. For this model the support for the difference between XY and ZW fusion rates was not as strong (92.0\% of the posterior probability supported $q_{XY.XY_F} > q_{ZW.ZW_F}$; Figure \ref{fig:squa-dif}).

\begin{figure}[p]
\centering
\includegraphics[scale=1.25]{figs/karyotype-fusion-squa-6par}
\caption[Fusion rate difference between XY and ZW systems (alternate model)]{Posterior estimate of the rate difference between XY and ZW fusions ($q_{XY.XY_F} - q_{ZW.ZW_F}$) in squamate reptiles when we allow the fission rates $q_{XY_F.XY}$ and $q_{ZW_F.ZW}$ to differ.}
\label{fig:squa-dif}
\end{figure}

As mentioned above, the squamate data contain very little information about fission rates, especially from $ZW_F$ to $ZW$. The likelihood approach has difficulty distinguishing between two explanations for the lack of fused ZW chromosomes: rare ZW fusions or common ZW fissions. Nevertheless, there is a strong signal that ZW fusions should be less common, which we confirmed by considering residency times $t_R$. For XY fusions,
\begin{equation}
t_{R,XY_F} = \frac{q_{XY.XY_F}}{q_{XY.XY_F} + q_{XY_F.XY}}
\end{equation}
and for ZW fusions
\begin{equation}
t_{R,ZW_F} = \frac{q_{ZW.ZW_F}}{q_{ZW.ZW_F} + q_{ZW_F.ZW}}
\end{equation}
Using a Bayesian analysis, we found very strong support for the residency time being greater for XY fusions than ZW fusions (99.8\% of the posterior probability supported $t_{R,XY_F} > t_{R,ZW_F}$; Figure \ref{fig:squa-resid}). In the absence of direct information about fission rates for fused ZW chromosomes, we conclude that the data is more parsimoniously explained by rare ZW fusions, while acknowledging that rapid ZW fission rates may also explain the data for squamates.

\begin{figure}[p]
\centering
\includegraphics[scale=1.1]{figs/karyotype-residency-squa-6par}
\caption[Fusion residency time in squamates]{Posterior estimate of the difference in residency time between XY and ZW fusions (i.e., $t_{R,XY_F} - t_{R,ZW_F}$) in squamate reptiles.}
\label{fig:squa-resid}
\end{figure}


\subsection{Comparing fusion rates between chromosomes} 

Rather than classifying the states as male/female heterogametic unfused/fused, we separated out the different types of fusions (e.g., classifying X-autosome [XA] and Y-autosome [YA] fusions as different states). This allowed us to assess whether the patterns we observed were driven by an overabundance of autosomal fusions with the Y chromosome. After matching the data to the tree, we did not have any records of WA fusions in fish while in squamates, XA fusions were absent. We thus considered models with only three fused states (for fish: XA, YA, and ZA; for squamates: YA, WA, and ZA)

For both the fish and the squamates, we again restricted the model via a nested series of likelihood ratio tests. For both clades, we found it to be statistically justifiable to assume that: a) transitions from one fused state directly to another fused state were impossible; b) prior to becoming fused, a lineage had to be in the corresponding unfused state; and c) fission rates were constrained to be equal. This allowed us to reliably evaluate whether the fusion rates differed by chromosome.

For the fish, using likelihood ratio tests, we found YA fusions to be significantly higher than XA fusions ($p=\text{0.016}$) and ZA fusions ($p=\text{0.035}$), but that XA and ZA fusion rates were not significantly different ($p=\text{0.658}$). Again, WA fusions did not exist in the fish analysis so we could not compare them to other classes. We then performed a Bayesian MCMC analysis to gain a better estimate of the relevant parameters. For the purposes of this analysis, we fixed XA and ZA fusions to occur at the same rate and then compared this rate to that for YA fusion. We found that YA fusions occur at a much higher rate than XA/ZA fusions (Figure \ref{fig:fish-ind}; 99.5\% of the posterior distribution supported this conclusion).

\begin{figure}[p]
\centering
\includegraphics[scale=1.25]{figs/chromosome-fusion-fish}
\caption[Comparison of Y-autosome and X-/Z-autosome fusion rates (fish)] {Posterior estimate of the rate difference between YA and XA/ZA fusions in fish. When the estimate is greater than zero, this means that the YA fusion rates are higher than those of the other chromosomes}
\label{fig:fish-ind}
\end{figure}

For the squamate analysis, YA fusions also occured at a higher rate than WA fusions ($p<\text{0.001}$) and ZA fusions ($p<\text{0.001}$). WA and ZA fusions rates were not significantly different from one another ($p\approx \text{1}$). As with the fish, for the Bayesian analysis we set WA and ZA fusion rates to be equal and estimated the difference between YA fusions and other type of fusions. 99.9\% of the posterior probability distribution supported YA fusions occuring at a higher rate than fusions on other chromosomes (Figure \ref{fig:squa-ind}). 

Taken together, these results strongly suggest that the difference between XY and ZW fusion rates is driven almost entirely by the very high rates of autosomal fusions involving the Y chromosome relative to the other sex chromosomes.

\begin{figure}[p]
\centering
\includegraphics[scale=1.25]{figs/chromosome-fusion-squa}
\caption[Comparison of Y-autosome and W-/Z-autosome fusion rates (squamates)]{Posterior estimate of the rate difference between YA and WA/ZA fusions in squamate reptiles. When the estimate is greater than zero, this means that the YA fusion rates are higher than those of the other chromosomes}
\label{fig:squa-ind}
\end{figure}

\section{Theoretical analysis}

To evaluate the plausibility of various mechanisms to explain the excess of fusions involving Y chromosomes, we modeled the rate of establishment of different sex chromosome-autosome fusions under various evolutionary scenarios. The core results are derived in \spacedsmallcaps{Appendix a}, where we approximate the rate, $R_C$, at which a given type of chromosome fusion $(C = X, Y, Z, \text{or} W)$ establishes within a population, accounting for both the rates that different types of fusions arise in a population and the probabilities that they fix.

To facilitate comparison to the data, we focus on the establishment rates for Y-A, Z-A, and W-A fusions relative to the rate of X-A fusions. We begin by studying the neutral case, where selection is absent. We allow, however, for sex-biased mutation rates and sex-biased sex ratios (see \spacedsmallcaps{Appendix a} for definitions). We then ask how these neutral results are altered by the three main evolutionary forces thought to impact the rate of fusions: direct selection acting on the fusion, meiotic drive, and sexually antagonistic selection.

\subsubsection{Neutral case}

We first consider the case without any selection or drive in the model. The overall establishment rates for fusions are given by the mutation rates generating each type of fusion (\spacedsmallcaps{Appendix a}, Equation \textsc{a.6}). Interestingly, the sex ratio does not enter into these results. Among newborns, each copy of a particular sex chromosome has an equal chance of being the progenitor of the entire population of that sex chromosome at some distant point in the future, regardless of subsequent changes in the survival and reproductive success of males versus females. 

Sex-biased mutation would alter the relative frequencies that different types of neutral fusions become fixed. Evidence suggests that the sexes differ substantially in the rate at which fusions arise: data from humans indicates that balanced translocations are the most likely source of new fusions \citep{Schubert2011}, and these seem to be predominantly paternal in origin \citep{Batista1993, Sartorelli2001, Wyrobek2006, Thomas2010, Grossmann2010, Schubert2011}. If mutation is male-biased but does not depend on the type of chromosome (that is, the X and Y chromosomes in a male are equally likely to fuse), then Y-A fusions will fix most frequently (see Equation \textsc{a.7}). In this case, however, Z-A fusions would be almost as common as Y-A fusions (at least 2/3 as common, see Equation \textsc{a.7} and Figure \ref{fig:fuse-direct}B, black curves), which is not seen in the data (Figures \ref{fig:phylo-fish} and \ref{fig:phylo-squa}). Thus the hypothesis that sex-autosome fusions are selectively neutral does not appear consistent with the data.

\subsubsection{Direct fitness effects}

We next ask how relative establishment rates depend on the direct fitness effects of a fusion (\spacedsmallcaps{Appendix a}). Assuming that the fusion has an additive effect on fitness and that all else is equal (unbiased reproductive sex ratios and mutation rates, and equal fitness effects for all types of fusions), the establishment rate is equal for X-A and Z-A fusions and for Y-A and W-A fusions (Figure \ref{fig:fuse-direct}). Fusions involving the Y or W are a factor $\frac{\text{1}}{\text{3}}(\text{1} + e^{-\text{2}N_s} + e^{-N_s})$
more common, where $N$ is the number of reproductive adults and $s$ is the fitness effect of the fusion. Thus, deleterious fusions ($s<\text{0}$) are much more likely to involve the Y or W chromosome, because of the smaller population size of these chromosomes (Figures \ref{fig:fuse-direct}A and \ref{fig:fuse-direct}C). Conversely, beneficial fusions are more likely to involve X or Z chromosomes because they are more numerous and so more often the targets of beneficial fusions (Figures \ref{fig:fuse-direct}B and \ref{fig:fuse-direct}D). 

Direct selection alone does not, however, explain why Y-A fusions are more common than W-A fusions. Similarly, direct selection cannot, on its own, explain why fusions in XY lineages are more common than in ZW lineages. To account for the observed data, therefore, we must invoke a combination of direct selection and sex biases, either in the sex ratio or in the mutation rate of fusions. 

Sexual selection is often stronger in males, which leads to a female-biased reproductive sex ratio \citep[that is, more reproducing females than males;][]{Bateman1948}. This situation will favor Y-A fusions over all other types if fusions are deleterious (Figure \ref{fig:fuse-direct}A). In this case, fusions are established by random genetic drift. Y fusions establish most frequently because the Y has the smallest effective population size of the four types of sex chromosomes because it is both hemizygous and restricted to the sex (males) with the fewest number of breeding individuals. By contrast, if fusions are beneficial, Y-A fusions are unlikely to be the most common type of fusion (only seen when there is an extremely male-biased sex ratio, with many fewer breeding females than males; see Equation \textsc{a.7} for very weak selection; Figure \ref{fig:fuse-direct}B). A second asymmetry that may be important to explaining the data is sex-biased mutation. As in the neutral case, we find that Y-A fusions will be most common when they are deleterious if they arise more often in males than females (blue, Figure \ref{fig:fuse-direct}B). These results strictly apply only when the fusion has an additive effect on fitness, but the relative frequencies of establishment for the different types of fusions are robust to changes in dominance.

\begin{figure}
\centering
\includegraphics[width=\textwidth]{figs/newerFig5}
\caption[Establishment rates of sex-autosome fusions under direct selection]{Establishment rates of sex-autosome fusions under direct selection, relative to the rate for X-A fusions. (A/B) Effect of sex ratio bias among reproductive adults, $N^m/(N^m + N^f)$, assuming $\mu^m=\mu^f$. (C/D) Effect of relative mutation rate for fusions in males versus females, $\mu^m/\mu^f$, assuming $N^f=N^m$. Mutations are deleterious ($s=-\text{0.0003}$) in panels (A) and (C), and beneficial ($s=\text{0.0003}$) in panels (B) and (D). Total effective population size $N^f+N^m=\text{10000}$.}
\label{fig:fuse-direct}
\end{figure}

Overall, selection acting against fusions combined with male-biased sex ratios and/or male-biased mutation rates can account for the observation that fusions in male heterogametic systems are substantially more common than in female heterogametic systems (Figure \ref{fig:pp-fuse-final}), and the observation that Y-A fusions are the most common (Figures. \ref{fig:fish-ind} and \ref{fig:squa-ind}). 

\subsubsection{Meiotic drive}
We next consider meiotic drive, which is thought to favor fused autosomes in some species of mammals and unfused chromosomes in others \citep{Pardo2001a, Pardo2001b}. If meiotic drive is weak, we can treat it as a form of direct selection, and so Equations \textsc{a.4} and \textsc{a.5} in \spacedsmallcaps{Appendix a} continue to apply. For clarity, we focus here on meiotic drive in females. (The results apply to meiotic drive in males if we interchange the sexes and the sex chromosomes, e.g., drive in ZW females becoming equivalent to drive in XY males.) Specifically, we assume that the probability that the fusion is transmitted to an egg is multiplied by a factor $1+f$ in fusion heterozygotes. If unfused chromosomes are preferentially transmitted to the egg, $f$ is negative. Averaging over the sexes, the effect of weak meiotic drive on an X-A fusion is equivalent to direct selection with a coefficient $s_X = \text{2}f/\text{3}$. (The factor of 2/3 appears because drive acts only when the fusion is in a female.) Thus when female meiotic drive favors fused chromosomes, the fixation probability for a single X-A fusion is higher than that for a Y-A fusion, which never experiences female meiotic drive (that is, it has an effective selection coefficient of $s_Y=\text{0}$). In ZW systems, a W-A fusion is always carried by females and so benefits in every generation when female meiosis is biased towards fused chromosomes ($s_W=f$), while Z-A fusions enjoy that advantage only one generation out of every three ($s_Z = f/\text{3}$).
 
Once we account for the numbers of each chromosome type (and assuming unbiased mutation rates and reproductive sex ratios), if female meiotic drive favors unfused chromosomes ($f < \text{0}$), then Y-A fusions are expected to establish at the highest rate, followed by W-A fusions, Z-A fusions, and last X-A fusions. The relative rankings are reversed if female meiotic drive favors fused chromosomes ($f > \text{0}$). Thus the observed excess of Y-A fusions can be accounted for by meiotic drive in females if unfused chromosomes benefit from drive relatively more often than fused chromosomes.

Meiotic drive in males could also account for a higher rate of Y-A fusions than X-A fusions if drive favors fusions, but under these conditions Z-A fusions would establish even more often (because there are three times as many Z chromosomes as Y chromosomes, and the Z spends 2/3 of its time in males). Thus, male meiotic drive alone cannot account for the excess of Y-A fusions over any other type of fusion, all else being equal.These effects of meiotic drive are robust to modest sex biases in mutation rates and the reproductive sex ratio. Large biases can, however, cause the relative order of establishment rates to switch in a manner that is qualitatively similar to that seen previously for fusions with direct fitness effects.

In sum, meiotic drive by itself does not seem a likely explanation for the observed excess of Y-A fusions. It could generate that pattern if drive acts in females and consistently favors unfused chromosomes. Data from mammals, however, suggest that when female meiotic drive acts on fusions, it sometimes favors fused but other times unfused chromosomes \citep{Pardo2001a, Pardo2001b}.

\subsubsection{Sexually antagonistic selection} 

To study fusions driven by sexually antagonistic selection, we developed a model that allows for sex-differences in selection (\spacedsmallcaps{Appendix a}). We assume that an autosomal locus segregates for alleles whose frequencies are at equilibrium before the fusion appears. (This equilibrium only occurs under some fitness values [\citealt{Clark1988}], and the following results apply only when those conditions are met.)

The fixation probability of a newly arisen fusion depends on several factors: which chromosome fuses with the autosome, whether the fusion originates in a male or a female, and which of the two alleles is captured by the fusion. The outcome also depends on the recombination rate in fused chromosomes between the sexually antagonistic locus and the sex-determining region; the models developed in \spacedsmallcaps{Appendix a} assume that linkage is complete. If drift is weak relative to selection, we find that fusions establish only if they are linked to the allele favoured in the sex in which the fused chromosome spends the most time, i.e., Y-A and Z-A fusions that capture a male-beneficial allele, and X-A and W-A fusions that capture a female-beneficial allele. 

Interestingly, if all else is equal (no sex biases in mutation rates or the reproductive sex ratio), the establishment rate of fusions is equal for all types of sex chromosomes (Equation \textsc{a.10}). Sex antagonistic selection tends to favour Y-A fusions and W-A fusions more strongly than X-A and Z-A fusions because these chromosomes are consistently found in a single sex \citep{Charlesworth1980}. This advantage, however, is exactly balanced by the lower rate that such fusions originate in the population because there are fewer Y and W chromosomes than X and Z chromosomes. Consequently, sexually antagonistic selection alone causes no difference in establishment rates.

To explain the observed excess of Y-A fusions by sexually antagonistic selection thus requires that the sexes differ in the mutation rate of fusions and/or in reproductive sex ratio (Equation \textsc{a.11}). Again, Y-A fusions will be particularly common if fusions originate more frequently in males. If the mutation rates are equal in males and females, however, then Y-A fusions will only be more common than X-A fusions if the reproductive sex ratio is male-biased (that is, more males than females reproduce), which is atypical. These conditions are illustrated in Figure \ref{fig:fuse-sa}. In general, if there is a combination of sex-biased mutation rates and biased sex ratios, Y-A fusions establish most frequently due to sexually antagonistic selection as long as $\mu^m N^m > \mu^fN^f$, where $\mu^m$ and $\mu^f$ are the female and male mutation rates, and $N^m$ and $N^f$ are the effective population sizes of females and males. When this condition is met, fusions also arise more often in XY lineages than in ZW lineages.

\begin{figure}
\centering
\includegraphics[width=\textwidth]{figs/newerFig6}
\caption[Establishment rates of sex-autosome fusions under SA selection]{Establishment rates of sex-autosome fusions as a result of sexually-antagonistic selection, relative to the rate for X-A fusions. The fusion is assumed to be neutral except for the effects of the sexually antagonistic allele that it captures. The fittest allele in each sex has a 10\% advantage when homozygous and a 9\% advantage when heterozygous (results are robust to these exact numbers). (A) Effect of sex ratio bias among reproductive adults, $N^m/(N^m + N^f)$, assuming $\mu^m=\mu^f$. (B) Effect of the relative mutation rate for fusions in males versus females, $\mu^f/\mu^m$ assuming $N^f=N^m$. Total effective population size $N^f+N^m=\text{10000}$.}
\label{fig:fuse-sa}
\end{figure}

\section{Discussion}

\subsection{Sex chromosome-autosome fusions are Y-biased in fishes and squamate reptiles}

A major finding in our study is that Y-autosome fusions occur more frequently than other sex chromosome fusions in vertebrates, particularly in fish and squamate reptiles. In amphibians, only one species in the database has multiple sex chromosomes, and it involves a Y-A fusion (Table \ref{tab:fusions}). Because mammals and birds have only male heterogametic and female heterogametic systems, respectively, we could not conduct phylogenetic tests to compare the relative rates of sex chromosome fusions involving XY versus ZW chromosomes. However, there are many mammalian species with Y-A fusions, whereas there are only three avian species with fusions, supporting our conclusion that Y-A fusions tend to occur more frequently than other fusions. 

Interestingly, mammals have roughly as many species with X-A fusions as with Y-A fusions, suggesting that the evolutionary forces acting on fusions may be different in mammals than in fish and reptiles. In particular, the form of female meiotic drive appears to vary among mammals, with drive favoring fused chromosomes in some species and unfused chromosomes in others, leading to a pattern in which species with X-A fusions tend to have metacentric chromosomes, while species with Y-A fusions tend to have acrocentric chromosomes \citep{Yoshida2012}.

Invertebrates provide a promising system for further phylogenetic analyses, with sex chromosome variation in several groups \citep{White1973, Bull1983, Charlesworth2005, ToS}. In Diptera there are seven ZW species and 986 XY species (plus 42 XO species) in the Tree of Sex database \citep{ToS}. Among these, there is a preponderance of fusions involving the Y: six Y-A fusions, one X-A fusion, and one species with both. Looking across all the invertebrates in the Tree of Sex database, there are many more cases of Y-A fusions (247 species) than X-A fusions (32 species), W-A fusions (8 species), and Z-A fusions (4 species); an additional 69 species have both X-A and Y-A fusions. While these data are consistent with the idea that Y-A fusions establish at a higher rate among invertebrates, a proper phylogenetic analysis is needed. A recent analysis of jumping spiders found only Y-A fusions (involving between four and seven independent events) among species that had both X and Y chromosomes \citep{White1973, Maddison2013}. Several X-A fusions were also identified, but these occurred only in species lacking a Y. Similar analyses in other groups of invertebrates promise to shed more light on sex chromosome evolution.

\subsection{Accounting for the high rate of Y-A fusions}

Our theoretical analyses clarify the conditions under which fusions involving the Y chromosome are more likely to establish. Interestingly, several plausible explanations fail to account for the data. Neutral fusions could account for an excess of Y-A over X-A fusions if fusions arise more often in males, but then the theory predicts that Z-A fusions should also be common, which contradicts the data (Table \ref{tab:fusions}, Figures \ref{fig:phylo-fish} and \ref{fig:phylo-squa}). Beneficial fusions also cannot explain the data, as they would tend to favor the accumulation of fusions involving the X or Z, which provide more abundant targets for new fusions than the Y or W. Furthermore, hypotheses in which fusions are established because they capture sexually antagonistic alleles also fail, because the higher fixation probability of Y-A fusions capturing male-beneficial alleles or W-A fusions capturing female beneficial alleles is exactly balanced by the lower population sizes of these sex chromosomes, decreasing the rate at which Y-A and W-A fusions enter a population. To account for the preponderance of Y-A fusions thus requires more complicated explanations, involving both selection and sex biases. We consider three plausible explanations below.

\subsubsection{Deleterious fusions with a sex biased mutation rate or reproductive sex ratio}

Chromosomal fusions may often have deleterious effects because fusions can lead to the loss of genetic material, alter gene expression, or impact the rate of segregation errors \citep{Ohno1967, Gardner2012}. Because the Y and W chromosomes have smaller effective population sizes than Z and W chromosomes, deleterious Y-A and W-A fusions are expected to fix more frequently than deleterious X-A and Z-A fusions.
To account for the excess of Y-A over W-A fusions, however, requires some sort of sex bias. One promising candidate is sexual selection, which often increases the variance in reproductive success of males relative to females \citep[Bateman's principle;][]{Bateman1948}. If fewer males are potentially successful as partners, the effective population size would be further reduced for the Y (but not for the W, carried by females) \citep{Bachtrog2011, Bandy2002}. As a consequence, we might expect Y-A fusions to be more frequent in polygynous mating systems (Figure \ref{fig:fuse-direct}A).

Another promising candidate is a male-biased mutation rate. Studies in humans suggest that chromosomal translocations, a common route to fusions, are more often of paternal origin than maternal \citep{Batista1993, Thomas2010, Grossmann2010}. By contrast, Robertsonian fusions (with two acrocentric chromosomes resulting in a fused metacentric chromosome) are more often maternal in origin \citep{Chamberlin1980, Bandy2002}, but this pattern may be confounded by female meiotic drive favoring the transmission of metacentric fusions in humans \citep{Pardo2001a}. While data from other species is needed, a preponderance of Y-A fusions can be explained if fusions are primarily deleterious and arise more often in males (Figure \ref{fig:fuse-direct}C). Of the three hypotheses we propose here, this may be the most compelling.

\subsubsection{Meiotic drive}

Because meiotic drive is often sex specific, it can break the symmetry between Y-A and W-A chromosomes and account for the high frequency of Y-A fusions. To do so requires female meiotic drive that selects against fused chromosomes, eliminating Z-A, W-A, and X-A fusions as they pass through female meiosis. Several cases of meiotic drive against fused chromosomes have been reported in mammals, for example in mice \citep{Pardo2001a, Pardo2001b}. On the other hand, female meiotic drive favors fused chromosomes in humans \citep{Pardo2001a}, while male meiotic favors fused chromosomes in the common shrew \citep{Searle1986, Wyttenbach1997}. Because the nature of meiotic drive varies among taxa, it seems unlikely that one particular form---female meiotic drive against fusions---is sufficiently widespread to explain the preponderance of Y-A fusions across vertebrates, particularly among fish (Figure \ref{fig:phylo-fish}) and squamate reptiles (Figure \ref{fig:phylo-squa}). Nevertheless, meiotic drive likely plays an important role in some taxa and may underlie the variation among mammals in rates of X-A and Y-A fusions \citep{Yoshida2012}.

\subsubsection{Sexually antagonistic selection with a sex biased mutation rate}

Sexually antagonistic selection is generally considered a key evolutionary factor in the turnover of sex chromosomes \citep{Charlesworth1980, vandoorn2007}. Our models, however, indicate that fusions involving the Y will be no more common than those involving other sex chromosomes once we account for the rate that Y fusions appear in the population and the fitness they gain by capturing a male-beneficial allele. In order to break the symmetry, we must again have either a male-biased mutation rate and/or a biased reproductive sex ratio. In this case, however, the sex ratio must be male biased, with less drift among males than females so that Y-A fusions establish more frequently than W-A fusions. Assuming that sexual selection typically generates the opposite sex ratio bias, with fewer breeding males than females, sexually-antagonistic selection requires even stronger male-biased mutation to explain the preponderance of Y-A fusions, compared to an explanation based on deleterious fusions. 

\subsection{Other considerations}

Other evolutionary forces not considered in this study may be important to the evolution of sex chromosome-autosome fusions. For example, we ignored inbreeding and spatial structure in our models. (We also did not consider fusions that capture alleles held polymorphic by heterozygote advantage, but the fate of fusions is unaffected by such loci [\citealt{Charlesworth1980}] unless there is inbreeding [\citealt{Charlesworth1999}].)
Furthermore, it is plausible that fusions may be more likely to involve some sex chromosomes for reasons that are independent of sex. For example, Y and W chromosomes often accumulate repetitive elements \citep{Bull1983, Charlesworth2005}, which could make them more prone to fusion through nonhomologous recombination. Alternatively, the Y and W may be less likely to be captured by a fusion when they are diminutive in size relative to the X and Z. Similarly, direct selection on fusions may be chromosome specific. For example, deletions and changes to gene expression may be less problematic on degenerated Y and Z chromosomes. While our analytical results allow for mutation rates and fitness effectsto depend on the specific chromosome involved (\spacedsmallcaps{Appendix a}), our figures and conclusions were drawn assuming that there were only sex-specific and not chromosome-specific effects. As more data emerge about chromosome-specific mutation rates and selection, the analytical results can guide refinements to these conclusions. 

\section{Concluding remarks}

Using phylogenetic analyses of fish and squamate reptiles, we show that fusions between sex chromosomes and autosomes more often involve the Y than other sex chromosomes. Using population genetic models, we find that this pattern cannot be explained by models of selection unless there is also some mechanism generating a difference between the sexes, including sex-biased mutation rates, biased sex ratios, or sex-specific selection (including meiotic drive). Perhaps the most plausible hypothesis to explain the data is that fusions occur more frequently in males, are slightly deleterious, and fix by drift. Similar factors may be important to the evolution of autosome-autosome fusions. If so, we expect autosomal fusions are also typically paternal origin in origin, deleterious, and established by drift.




